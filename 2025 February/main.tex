\documentclass{beamer}

\usetheme[secheader]{Madrid}

\usepackage{graphicx}
\graphicspath{ {./images/} }
\usepackage[table]{xcolor}


\usepackage{tikz}
\usetikzlibrary{shapes,arrows}
\usepackage[edges]{forest}
\usetikzlibrary{trees}

\usepackage{pifont}

\newcommand{\highlight}[1]{{\color{blue} #1}}
\newcommand{\pending}[1]{{\color{red} #1}}
\newcommand{\done}{{\color{green}\ding{52}}}
\newcommand{\todo}{{\color{red}\ding{56}}}
\newcommand{\doing}{{\color{orange}\ding{229}}}
\newcommand{\new}{{\colorbox{blue!30}{\textcolor{white}{\textbf{\scriptsize New!}}}}}
\newcommand{\goal}{{\colorbox{green!30}{\textcolor{white}{\textbf{\scriptsize Goal!}}}}}
\newcommand{\focus}{{\colorbox{red!30}{\textcolor{white}{\textbf{\tiny !}}}}}

\title{Primera Reunión General del LMRI}
\subtitle{TIC: Unidad de Tecnologías de la Información y el Conocimiento}
\author[X. Campo]{Xandra Campo}
\institute[LMRI-CIEMAT]{Laboratorio de Metrología de Radiaciones Ionizantes (LMRI) \newline CIEMAT}
\date{13 de febrero de 2025}<

\begin{document}

	\maketitle

	\begin{frame}
		\frametitle{Table of Contents}
		\tableofcontents
	\end{frame}

	\section{Unidad de TIC}

	\subsection{Proyectos}

	\subsubsection{IR14: EURAMET GuideRadPROS project}

	\begin{frame}
		\frametitle{Proyectos}
		\framesubtitle{IR14: EURAMET GuideRadPROS project}
		\begin{columns}
			\column{0.44\textwidth}
				\begin{itemize}
					\item \highlight{Pagina web proyecto} \done
					\begin{itemize}
						\item Prácticas FP  \done
					\end{itemize}
					\item \highlight{Librería USpekPy} \doing
					\begin{itemize}
						\item Aplicación web \doing
						\begin{itemize}
							\item Prácticas FP \done
						\end{itemize}
						\item Seminario de uso \done
						\item Script ficheros entrada \done
						\item Script análisis librería \done
						\item Publicar SpekPy \done
					\end{itemize}
				\end{itemize}
			\column{0.56\textwidth}
				\begin{itemize}
					\item \highlight{Análisis de datos} \doing
					\begin{itemize}
						\item Script espectrometría \done
						\item Script medida de HVL \todo
						\item Script gráficas para análisis \todo
						\item Script caracterización haces \new
					\end{itemize}
					\item \highlight{Otros}
					\begin{itemize}
						\item Estancia Joonas Tikkanen, STUK (¿verano 2025?) \new
						\item Curso Calibrating photon dosimeters to ISO 4037 standards (20-22 mayo, Atenas) \new
					\end{itemize}
				\end{itemize}
		\end{columns}

	\end{frame}

	\subsubsection{IR14-D: Patrones dosimétricos de rayos X}

	\begin{frame}
		\frametitle{Proyectos}
		\framesubtitle{IR14-D: Patrones dosimétricos de rayos X}
		\begin{itemize}
			\item \highlight{Automatización cadena de medida} \doing \ \textbf{?}
			\begin{itemize}
				\item Aplicación de escritorio calibración \doing
				\item Librería MetPyX \doing
			\end{itemize}
			\item \highlight{Aplicación lectura barómetro} \done
			\item \highlight{Scripts} \doing
			\begin{itemize}
				\item Espectrometría experimental \done
				\item Medida de HVL \todo
				\item Calibración \doing
				\item Asignación de dosis \todo
			\end{itemize}
			\item \highlight{Librería MetPyX} \doing \ \textbf{?}
		\end{itemize}
	\end{frame}

	\subsubsection{LMRI}

	\begin{frame}
		\frametitle{Proyectos}
		\framesubtitle{LMRI}
		\begin{itemize}
			\item \highlight{Organización GitHub} \done
			\item \highlight{Servidores LMRI} \doing \ \textbf{?}
			\begin{itemize}
				\item Interno \doing
				\item Externo \doing
			\end{itemize}
			\item \highlight{Librería MetPy} \doing \ \textbf{?}
			\begin{itemize}
				\item Calculo de incertidumbres \todo
				\item Interpolador \doing
			\end{itemize}
			\item \highlight{Curso ecosistema de trabajo de Python} \todo
			\item Página web LMRI
			\begin{itemize}
				\item \highlight{Modernización} de la web del LMRI \todo
				\item Aplicación web solicitud de \highlight{servicios técnicos} \doing
			\end{itemize}
		\end{itemize}
	\end{frame}

	\subsubsection{IR 13: Metrología de radionucleidos}

	\begin{frame}
		\frametitle{Proyectos}
		\framesubtitle{IR 13: Metrología de radionucleidos}
		\begin{itemize}
			\item \highlight{Librería MetPyRad} \doing \ \new
			\begin{itemize}
				\item Procesado ficheros salida Hidex 300 SL \doing
				\item Procesado ficheros salida contador de VP \todo
				\item Procesado ficheros salida prototipo TDCR \todo
				\item Calculo periodo de semidesintegración desde ficheros \todo
			\end{itemize}
			\item \highlight{Aplicación de escritorio protopipo TDCR} \doing \ \new
		\end{itemize}
	\end{frame}

	\subsubsection{Otros}

	\begin{frame}
		\frametitle{Proyectos}
		\framesubtitle{Otros}
		\begin{itemize}
			\item \highlight{Reuniones} \new
			\begin{itemize}
				\item EURAMET Spring Camp on Digitalisation (25-28 marzo, Bratislava)
			\end{itemize}
			\item \highlight{Formación} \new
			\begin{itemize}
				\item Curso INAP Programación web (17 febrero - 7 abril, online)
				\item Curso Prevención y actuación temprana frente al acoso sexual y por razón de sexo, incluido en el ámbito digital (20-24 febreo, online)
			\end{itemize}
		\end{itemize}
	\end{frame}

	\subsubsection{Enlaces de interés}
	
	\begin{frame}
		\frametitle{Proyectos}
		\framesubtitle{Herramientas públicas: enlaces de interés}
		\centering
		\scriptsize
		\rowcolors{2}{gray!15}{white}
		\begin{tabular}{ll}			
			\rowcolor{blue!40}
			{\color{white}GuideRadPROS}&\\
			Web del proyecto&https://github.com/lmri-met/sites-guideradpros\\
			&https://lmri-met.github.io/sites-guideradpros/\\
			USpekPy: Librería&https://github.com/lmri-met/uspekpy\\
			USpekPy: Seminario&https://github.com/xandratxan/uspekpy-seminar\\
			USpekPy: Análisis librería&https://github.com/xandratxan/using-uspekpy\\
			USpekPy: Generador input&https://github.com/xandratxan/uspekpy-input-generator\\
			USpekPy: Aplicación web&https://github.com/lmri-met/uspekpy-web\\
			SpekPy: Librería&https://pypi.org/project/spekpy/\\
			\rowcolor{blue!40}
			{\color{white}IR14-D}&\\
			MetPyX: Librería&https://github.com/lmri-met/metpyx\\
			\rowcolor{blue!40}
			{\color{white}IR13}&\\
			MetPyRad: Librería&https://github.com/lmri-met/metpyrad\\
			\rowcolor{blue!40}
			{\color{white}LMRI}&\\
			Organización del LMRI en GitHub&https://github.com/lmri-met\\
			Librería incertidumbres&https://github.com/xandratxan/physical-magnitude\\
		\end{tabular}
	\end{frame}
	
	\subsection{Objetivos}

	\begin{frame}
		\frametitle{Objetivos}
		\framesubtitle{Segundo semestre 2024 y primer cuatrimestre 2025}
		\centering
		\scriptsize
		\rowcolors{2}{gray!15}{white}
		\begin{tabular}{lcccc}
			&S1 2024&S2 2024&C1 2025&C2 2025\\
			\rowcolor{blue!40}
			{\color{white}GuideRadPROS}&&&&\\
			USpekPy: Aplicación web                    &\doing&\doing&\doing&\doing\\
			Análisis: Script medida de HVL             &      &      &\todo &\\
			Análisis: Script gráficas para análisis    &      &      &\todo &\\
			Análisis: Caracterización haces \focus     &      &      &\todo &\\
			\rowcolor{blue!40}
			{\color{white}IR14-D}&&&&\\
			Aplicación escritorio calibración&\doing&\doing&\textbf{?}&\textbf{?}\\
			Librería MetPyX                  &\doing&\doing&\textbf{?}&\textbf{?}\\
			\rowcolor{blue!40}
			{\color{white}IR13}&&&&\\
			MetPyRad: Soporte Hidex 300 SL \focus          &&&\doing&\\
			MetPyRad: Soporte equipo VP \focus             &&&\todo &\doing\\
			MetPyRad: Soporte cálculo periodo \focus       &&&\todo &\doing\\
			Aplicación de escritorio prototipo TDCR \focus &&&\doing&\doing\\
			\rowcolor{blue!40}
			{\color{white}LMRI}&&&&\\
			Servidor web externo   &\doing&\doing&\doing    &\doing\\
			Servidor web interno   &\doing&\doing&\textbf{?}&\textbf{?}\\
			Curso ecosistema Python&      &\todo &\textbf{?}&\textbf{?}\\
		\end{tabular}
	\end{frame}
	
	\subsection{Necesidades y propuestas}
	
	\begin{frame}
		\frametitle{Necesidades y propuestas}
		\highlight{Propuestas}
		\begin{itemize}
			\item Validación de hojas de cálculo de calibración y/o asignación de dosis con scripts de Python
		\end{itemize}
		\highlight{Necesidades}
		\begin{itemize}
			\item Ordenador + periféricos para servidor interno \done
		\end{itemize}
	\end{frame}
	
	\subsection{}
	
	\begin{frame}
		\begin{block}{}
			\centering
			¡Gracias por vuestra atención!
		\end{block}
	\end{frame}

\end{document}